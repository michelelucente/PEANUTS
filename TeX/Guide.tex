\documentclass{article}

\usepackage{amsmath}
\usepackage{hyperref}
\usepackage{graphicx}
\graphicspath{ {./figs/} }

\newcommand{\MeV}[0]{\text{MeV}}
\newcommand{\cm}[0]{\text{cm}}
\newcommand{\de}[0]{\text{d}}

\begin{document}

\section{Solar neutrino flux}
	\subsection{Neutrino survival probability at Sun surface}
	
	The survival probability at the exit of the Sun for an electron neutrino produced with energy $E$ is given, under the adiabatic approximation, by (cf. Eq. 6.14 in \cite{FiuzadeBarros:2011qna})
	
	\begin{equation}
		P_{\nu_e \rightarrow \nu_e}(E, n_e(x)) = c_{13}^2 c_{13,M}^2 \left( c_{12}^2 c_{12,M}^2 + s_{12}^2 s_{12,M}^2 \right) + s_{13}^2 s_{13,M}^2,
	\end{equation}
	
where $c_{ij} \equiv \cos \theta_{ij},\ s_{ij} \equiv \sin \theta_{ij}$ and the mixing angles in matter $\theta_{ij}^M$ are related to the vacuum ones by

\begin{eqnarray}
	\tan 2 \theta_{12}^M &=& \frac{\tan 2 \theta_{12}}{1-\frac{\cos^2 \theta_{13} A_{CC}}{\cos \theta_{12} \Delta m^2_{21}} },\\
	\sin \theta_{13}^M &=& \sin \theta_{13} \left[1 + \frac{A_{CC}}{\Delta m^2_{31}} \cos^2\theta_{13}\right],
\end{eqnarray}
where
\begin{equation}
	A_{CC} = 2 E V_{CC} = 2 \sqrt{2} E G_F n_e(x),
\end{equation}
represents the matter potential, with $E$ the neutrino energy, $G_F$ the Fermi constant and $n_e(x)$ the electron density at the neutrino creation point $x$.

Numerically (cf. eq.s 4.17, 4.18 in \cite{Fantini:2018itu})
\begin{eqnarray}
	V &=& \sqrt{2} G_F n_e = \frac{3.868 \times 10^{-7}}{m} \times \frac{n_e}{\text{mol}/\text{cm}^3},\\
	k &=& \frac{\Delta m^2}{2 E} = \frac{2.533}{m} \times \frac{\Delta m^2}{\text{eV}^2} \times \frac{\MeV}{E},
\end{eqnarray}
where the wavenumber $k$ is useful in the relation
\begin{equation}
	\frac{A_{CC}}{\Delta m^2} = \frac{2 \sqrt{2} E G_F n_e}{\Delta m^2} = \frac{V}{k}.
\end{equation}

For a fixed value of neutrino energy $E$, the survival probability at Sun surface is given by the average over the neutrino production point inside the Sun. Assuming spherical symmetry, if $f(r)$ is the fraction of neutrinos produced at point $r \equiv R/R_\odot$, where $R_\odot$ is the solar radius and $R$ the distance from the center of the Sun, 
\begin{equation}
	P_{\nu_e \rightarrow \nu_e}(E) = \int_0^1 \de r P_{\nu_e \rightarrow \nu_e}(E, n_e(r)) f(r),
\end{equation}
with the normalization
\begin{equation}
	\int_0^1 \de r f(r) = 1.
\end{equation}

To compute $f(r)$ we need a solar model. In a first implementation we use the BS05(AGS,OP) model from~\cite{Bahcall:2004pz} (data from \url{http://www.sns.ias.edu/~jnb/}; \emph{note that this file must be processed before importing, since both since and double whitespaces are used as separators of the tabulated data}), since this is the model used by the SNO collaboration. Once the code will be stable we will easily be able to update to a more recent model.

The electron density in~\cite{Bahcall:2004pz} is given in units:
\begin{equation}
	\frac{1}{\cm^3 N_A} = \frac{\text{mol}}{\cm^3\ 6.022 \cdot 10^{23} }.
\end{equation}
	
\begin{figure}[htb]
\centering
\includegraphics[width=0.8\textwidth]{sun_density.pdf}
\caption{Solar density as a function of relative radius $r \equiv R/R_\odot$, from Solar model BS05(AGS,OP)~\cite{Bahcall:2004pz}.}
\end{figure}

\begin{figure}
	\includegraphics[width=0.8\textwidth]{reaction_fraction.pdf}
	\centering
	\caption{Fraction of solar neutrinos produced as function of distance from the center of the Sun $r\equiv R/R_\odot$, from Solar model BS05(AGS,OP)~\cite{Bahcall:2004pz}.}
\end{figure}

\bibliographystyle{unsrt}
\bibliography{bibliography.bib}

\end{document}