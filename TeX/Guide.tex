\documentclass{article}

\usepackage{amsmath}
\usepackage{hyperref}
\usepackage{graphicx}
\graphicspath{ {./figs/} }
\usepackage{bbold}

\newcommand{\MeV}[0]{\text{MeV}}
\newcommand{\cm}[0]{\text{cm}}
\newcommand{\de}[0]{\text{d}}
\newcommand{\ket}[1]{\left| #1 \right>}
\newcommand{\bra}[1]{\left< #1 \right|}
\newcommand{\braket}[2]{\left< #1 | #2 \right>}

\begin{document}

\section{Solar neutrino flux}
	\subsection{Neutrino survival probability at Sun surface}
	
	The survival probability at the exit of the Sun for an electron neutrino produced with energy $E$ is given, under the adiabatic approximation, by (cf. Eq. 6.14 in \cite{FiuzadeBarros:2011qna})
	
	\begin{equation}
		P_{\nu_e \rightarrow \nu_e}(E, n_e(x)) = c_{13}^2 c_{13,M}^2 \left( c_{12}^2 c_{12,M}^2 + s_{12}^2 s_{12,M}^2 \right) + s_{13}^2 s_{13,M}^2,
	\end{equation}
	
where $c_{ij} \equiv \cos \theta_{ij},\ s_{ij} \equiv \sin \theta_{ij}$ and the mixing angles in matter $\theta_{ij}^M$ are related to the vacuum ones by

\begin{eqnarray}
	\tan 2 \theta_{12}^M &=& \frac{\tan 2 \theta_{12}}{1-\frac{\cos^2 \theta_{13} A_{CC}}{\cos \theta_{12} \Delta m^2_{21}} },\\
	\sin \theta_{13}^M &=& \sin \theta_{13} \left[1 + \frac{A_{CC}}{\Delta m^2_{31}} \cos^2\theta_{13}\right],
\end{eqnarray}
where
\begin{equation}
	A_{CC} = 2 E V_{CC} = 2 \sqrt{2} E G_F n_e(x),
\end{equation}
represents the matter potential, with $E$ the neutrino energy, $G_F$ the Fermi constant and $n_e(x)$ the electron density at the neutrino creation point $x$.

Numerically (cf. eq.s 4.17, 4.18 in \cite{Fantini:2018itu})
\begin{eqnarray}
	V &=& \sqrt{2} G_F n_e = \frac{3.868 \times 10^{-7}}{m} \times \frac{n_e}{\text{mol}/\text{cm}^3},\\
	k &=& \frac{\Delta m^2}{2 E} = \frac{2.533}{m} \times \frac{\Delta m^2}{\text{eV}^2} \times \frac{\MeV}{E},
\end{eqnarray}
where the wavenumber $k$ is useful in the relation
\begin{equation}
	\frac{A_{CC}}{\Delta m^2} = \frac{2 \sqrt{2} E G_F n_e}{\Delta m^2} = \frac{V}{k}.
\end{equation}

For a fixed value of neutrino energy $E$, the survival probability at Sun surface is given by the average over the neutrino production point inside the Sun. Assuming spherical symmetry, if $f(r)$ is the fraction of neutrinos produced at point $r \equiv R/R_\odot$, where $R_\odot$ is the solar radius and $R$ the distance from the center of the Sun, 
\begin{equation}
	P_{\nu_e \rightarrow \nu_e}(E) = \int_0^1 \de r P_{\nu_e \rightarrow \nu_e}(E, n_e(r)) f(r),
\end{equation}
with the normalization
\begin{equation}
	\int_0^1 \de r f(r) = 1.
\end{equation}

To compute $f(r)$ we need a solar model. In a first implementation we use the BS05(AGS,OP) model from~\cite{Bahcall:2004pz} (data from \url{http://www.sns.ias.edu/~jnb/}; \emph{note that this file must be processed before importing, since both since and double whitespaces are used as separators of the tabulated data}), since this is the model used by the SNO collaboration. Once the code will be stable we will easily be able to update to a more recent model.

The electron density in~\cite{Bahcall:2004pz} is given in units:
\begin{equation}
	\frac{1}{\cm^3 N_A} = \frac{\text{mol}}{\cm^3\ 6.022 \cdot 10^{23} }.
\end{equation}
	
\begin{figure}[htb]
\includegraphics[width=0.5\textwidth]{sun_density.pdf}
\includegraphics[width=0.5\textwidth]{reaction_fraction.pdf}
\caption{Solar density (left) and fraction of solar neutrinos produced (right), as a function of relative radius $r \equiv R/R_\odot$, from Solar model BS05(AGS,OP)~\cite{Bahcall:2004pz}.}
\end{figure}

We compare in Fig.~\ref{fig:SNO_8B_hep_comparison} the prediction from our code with the SNO survival probabilities at the surface of the Sun for the ${}^8$B and hep neutrino, digitised from Fig. 6.3 in~\cite{FiuzadeBarros:2011qna}. 

\begin{figure}[htb]
	\includegraphics[width=0.5\textwidth]{8B_SNO_cmparison}
	\includegraphics[width=0.5\textwidth]{hep_SNO_comparison}
	\caption{Survival probability at the surface of the Sun for mixing parameters $\tan^2\theta_{12} = 0.469$, $\Delta m_{21}^2 = 7.9 \times 10^{-5} \text{eV}^2$, $\sin^2 \theta_{13} = 0.01$, $\Delta m_{31}^2 = 2.46 \times 10^{-3} \text{eV}^2$. The blue lines are the prediction from our code, the orange ones are the digitised curves from 6.3 in~\cite{FiuzadeBarros:2011qna}.}
	\label{fig:SNO_8B_hep_comparison}
\end{figure}

\section{Neutrino propagation from Sun and detection at Earth}

\subsection{Probability of transition $\nu_i \rightarrow \nu_\alpha$}
The neutrino interaction (aka flavour) eigenstates $\ket{\nu_\alpha, t}$ at time $t$ are related to the propagation (aka mass) eigenstates by the relation
\begin{equation}
	\ket{\nu_\alpha, t} = U_{\alpha i}^* \ket{\nu_i, t}
\end{equation}
where $U$ is the unitary PMNS mixing matrix. The flavour eigenstate at time $t$ is  obtained by evolving the one at $t=0$ with the evolutor operator
\begin{equation}
	\ket{\nu_\alpha, t} = \mathcal{U}_{\alpha \beta}(t) \ket{\nu_\beta,0}.
\end{equation}
From this relation the probability amplitude for a neutrino mass eigenstate produced at $t=0$ of interacting as a flavour eigenstate at time $t$ is given by
\begin{eqnarray}
	P_{i \rightarrow \alpha}(t) &=& \braket{\nu_\alpha, 0}{\nu_i,t} = \bra{\nu_\alpha, 0} U^T_{i \beta} \ \mathcal{U}_{\beta \gamma}(t) \ket{\nu_\alpha,0} \\ 
	&=& U^T_{i\beta} \ \mathcal{U}_{\beta \alpha}(t) = \left( U^T \mathcal{U}(t) \right)_{i\alpha}
\end{eqnarray}

\subsection{Evolutor equation and Magnus expansion}

The evolutor operator is defined by the equations
\begin{equation}
	\ket{\nu_\alpha, t} = \mathcal{U}_{\alpha \beta}(t) \ket{\nu_\beta, 0}, \hspace{1cm} \text{ with } \hspace{1cm} \mathcal{U}(0) = \mathbb{1}.
\end{equation}

From the Schr{\"o}dinger equation
\begin{eqnarray}
	\frac{\de}{\de t} \ket{\nu_a\alpha, t}  &=& - i H_{\alpha \beta} \ket{\nu_\beta, t} \\
	\Rightarrow \hspace{0.2cm} \frac{d}{\de t} \mathcal{U}_{\alpha \beta} (t) \ket{\nu_\beta, 0} &=& -i H_{\alpha \gamma}(t)\ \mathcal{U}_{\gamma \beta}(t) \ket{\nu_\beta, 0} \\
	\Rightarrow \hspace{0.2cm} \frac{\de}{\de t} \mathcal{U}(t) &=& - i H(t)\ \mathcal{U}(t), \label{eq:evolutor}
\end{eqnarray}
where $H$ is the Hamiltonian of the system.

Eq.~\ref{eq:evolutor} allows to express the evolutor as a function of the Hamiltonian of the system. Its formal solution is
\begin{equation}
	\mathcal{U}(t) = \mathcal{T} \left[ e^{-i \int_{t_0}^t \de t' H(t')} \right].\label{eq:evolutor_formal}
\end{equation}
Eq.~\ref{eq:evolutor_formal} does not generally admit an analytic closed form, expect for very special cases, for instance if the Hamiltonian does not depend on time.

A well known approach to the problem is the Dyson series 
\begin{equation}
	\mathcal{U}(t) = \mathbb{1} + \sum_{n=1}^\infty \frac{\left(-i\right)^n}{n!} \int_{t_0}^t \de t_1 \int_{t_0}^t \de t_2 \cdots \int_{t_0}^t \de t_n \mathcal{T} \left[H(t_1) H(t_2) \cdots H(t_n) \right] \label{eq:Dyson_series}
\end{equation}
which allows for an approximate solution obtained by truncating Eq.~\ref{eq:Dyson_series} at finite values of $n$, if the series is expected to be perturbative. However, by adopting this approach the approximate evolutor operator is not longer a unitary operator.

An alternative approximate solution, that we adopt here, is the Magnus expansion~\cite{magnus1954exponential} (ME) (see also~\cite{blanes2009magnus} for a modern review).
Given the matrix differential equation
\begin{equation}
	\frac{\de}{\de t} Y(t) = A(t) Y(t), \hspace{1cm} \text{with} \hspace{1cm} Y(0) = Y_0,
\end{equation}
the solution at time $t$ can be expressed as
\begin{equation}
	Y(t) = e^{\Omega(t)} Y_0, \hspace{1cm} \text{with} \hspace{1cm} \Omega(t) = \sum_{k=1}^\infty \Omega_k (t). \label{eq:magnus_formal}
\end{equation}
The operators $\Omega_k(t)$ are formally defined for each value of $k$. As with the Dyson series, Eq.~\ref{eq:magnus_formal} gives a formal solution, but for practical applications the series must be truncated at finite values of $k$. Nevertheless, the approximate evolutor operator is unitary in the ME.

The first two terms in the ME are
\begin{eqnarray}
	\Omega_1(t) &=& \int_0^t \de t_1 A(t_1), \\
	\Omega_2 (t) &=& \frac{1}{2} \int_0^t \de t_1 \int_0^{t_1}  \de t_2 \left[ A(t_1), A (t_2) \right].
\end{eqnarray} 
$\Omega_1$ would give the exact solution if $A$ is constant, such that its commutator at different times vanish. For the general case, the higher-order terms in the ME take into account the $A(t)$ time dependency.

In the neutrino oscillation dynamics the starting point is Eq.~\ref{eq:evolutor}, with ${\mathcal{U}(0) = \mathbb{1}}$. The ME of the evolutor is thus
\begin{equation}
	\mathcal{U}(t) = e^{\Omega(t)},\hspace{1cm} \text{with} \hspace{1cm} \Omega(t) = \sum_{k=1}^\infty \Omega_k (t)
\end{equation}

In the adiabatic regime solar neutrinos leave the Sun as propagation eigenstates $\ket{\nu_2}$.

\bibliographystyle{unsrt}
\bibliography{bibliography.bib}

\end{document}