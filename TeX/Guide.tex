\documentclass{article}

\usepackage{amsmath}
\usepackage{hyperref}
\usepackage{graphicx}
\graphicspath{ {./figs/} }
\usepackage{bbold}
\usepackage{color}

\newcommand{\todo}[1]{\textbf{To do:} \emph{\color{blue} #1}}
\newcommand{\MeV}[0]{\text{MeV}}
\newcommand{\cm}[0]{\text{cm}}
\newcommand{\de}[0]{\text{d}}
\newcommand{\ket}[1]{\left| #1 \right>}
\newcommand{\bra}[1]{\left< #1 \right|}
\newcommand{\braket}[2]{\left< #1 | #2 \right>}

\begin{document}

\section{Solar neutrino flux}
	\subsection{Neutrino survival probability at Sun surface}
The solar neutrino flux at Earth is given by an incoherent superposition of neutrino mass eigenstates~\cite{Mikheev:1987wa, Lisi:1997yc}. In the adiabatic approximation the neutrino state at the Sun surface only depends on the matter effects at neutrino production point; being $T$ the matrix that diagonalises the neutrino Hamiltonian in matter~\cite{FiuzadeBarros:2011qna}: \todo{Check diagonalisation, probably CP-phases missing - does not impact since we square}
\begin{equation}
	T\left(E, n_e\right) =\left( \begin{array}{ccc}
		c_{13, M} c_{12, M} & c_{13,M} s_{12, M} & s_{13, M} \\
		-s_{12, M} & c_{12,M} & 0 \\
		-s_{13,M} c_{12,M} & -s_{13,M} s_{12,M} & c_{13, M}
	\end{array}
	\right),
\end{equation}
	
%	The survival probability at the exit of the Sun for an electron neutrino produced with energy $E$ is given, under the adiabatic approximation, by (cf. Eq. 6.14 in \cite{FiuzadeBarros:2011qna})
%	
%	\begin{equation}
%		P_{\nu_e \rightarrow \nu_e}(E, n_e(x)) = c_{13}^2 c_{13,M}^2 \left( c_{12}^2 c_{12,M}^2 + s_{12}^2 s_{12,M}^2 \right) + s_{13}^2 s_{13,M}^2,
%	\end{equation}
	
where $c_{ij} \equiv \cos \theta_{ij},\ s_{ij} \equiv \sin \theta_{ij}$ and the mixing angles in matter $\theta_{ij}^M$ are related to the vacuum ones by

\begin{eqnarray}
	\tan 2 \theta_{12}^M &=& \frac{\tan 2 \theta_{12}}{1-\frac{\cos^2 \theta_{13} A_{CC}}{\cos \theta_{12} \Delta m^2_{21}} },\\
	\sin \theta_{13}^M &=& \sin \theta_{13} \left[1 + \frac{A_{CC}}{\Delta m^2_{31}} \cos^2\theta_{13}\right],
\end{eqnarray}
whit
\begin{equation}
	A_{CC} = 2 E V_{CC} = 2 \sqrt{2} E G_F n_e(x),
\end{equation}
representing the matter potential,  $E$ the neutrino energy, $G_F$ the Fermi constant and $n_e(x)$ the electron density at the neutrino creation point $x$.

Numerically (cf. eq.s 4.17, 4.18 in \cite{Fantini:2018itu})
\begin{eqnarray}
	V &=& \sqrt{2} G_F n_e = \frac{3.868 \times 10^{-7}}{m} \times \frac{n_e}{\text{mol}/\text{cm}^3},\\
	k &=& \frac{\Delta m^2}{2 E} = \frac{2.533}{m} \times \frac{\Delta m^2}{\text{eV}^2} \times \frac{\MeV}{E},
\end{eqnarray}
where the wavenumber $k$ is useful in the relation
\begin{equation}
	\frac{A_{CC}}{\Delta m^2} = \frac{2 \sqrt{2} E G_F n_e}{\Delta m^2} = \frac{V}{k}.
\end{equation}

The probability of producing a neutrino mass eigenstate $i$ in matter is thus $\left| T_{\alpha i}\left( E, n_e \right)\right|^2$, where $\alpha$ is the flavour of the charged lepton entering the neutrino production vertex.

In the adiabatic approximation neutrinos evolve as pure mass eigenstates within the Sun. For a fixed value of neutrino energy $E$, the flux composition  at Sun surface is given by the average over the neutrino production points inside the Sun. Assuming spherical symmetry, if $f(r)$ is the fraction of neutrinos produced at point $r \equiv R/R_\odot$, where $R_\odot$ is the solar radius and $R$ the distance from the center of the Sun, 
\begin{equation}
	P_{\nu_e \rightarrow \nu_i}^\odot(E) = \int_0^1 \de r \left| T_{e i}\left( E, n_e(r) \right)\right|^2 f(r),
\end{equation}
with the normalization
\begin{equation}
	\int_0^1 \de r f(r) = 1.
\end{equation}

To compute $f(r)$ we need a solar model. In a first implementation we use the BS05(AGS,OP) model from~\cite{Bahcall:2004pz} (data from \url{http://www.sns.ias.edu/~jnb/}; \emph{note that this file must be processed before importing, since both since and double whitespaces are used as separators of the tabulated data}), since this is the model used by the SNO collaboration. Once the code will be stable we will easily be able to update to a more recent model.

The electron density in~\cite{Bahcall:2004pz} is given in units:
\begin{equation}
	\frac{1}{\cm^3 N_A} = \frac{\text{mol}}{\cm^3\ 6.022 \cdot 10^{23} }.
\end{equation}
	
\begin{figure}[ht]
\includegraphics[width=0.5\textwidth]{sun_density.pdf}
\includegraphics[width=0.5\textwidth]{reaction_fraction.pdf}
\caption{Solar density (left) and fraction of solar neutrinos produced (right), as a function of relative radius $r \equiv R/R_\odot$, from Solar model BS05(AGS,OP)~\cite{Bahcall:2004pz}.}
\end{figure}

We compare in Fig.~\ref{fig:SNO_8B_hep_comparison} the prediction from our code with the SNO survival probabilities at the surface of the Sun for the ${}^8$B and hep neutrino, digitised from Fig. 6.3 in~\cite{FiuzadeBarros:2011qna}. \todo{The SNO reference uses different values than current best fits. Prepare model files for it.}

\begin{figure}[ht]
	\includegraphics[width=0.5\textwidth]{8B_SNO_comparison}
	\includegraphics[width=0.5\textwidth]{hep_SNO_comparison}
	\caption{Survival probability at the surface of the Sun for mixing parameters $\tan^2\theta_{12} = 0.469$, $\Delta m_{21}^2 = 7.9 \times 10^{-5} \text{eV}^2$, $\sin^2 \theta_{13} = 0.01$, $\Delta m_{31}^2 = 2.46 \times 10^{-3} \text{eV}^2$. The blue lines are the prediction from our code, the orange ones are the digitised curves from 6.3 in~\cite{FiuzadeBarros:2011qna}.}
	\label{fig:SNO_8B_hep_comparison}
\end{figure}

\section{Neutrino propagation from Sun and detection at Earth}

\subsection{Probability of transition $\nu_i \rightarrow \nu_\alpha$}
Being it an incoherent flux, the fraction of mass eigenstates within the solar neutrino flux remains constant as long as neutrinos propagate in the vacuum, on their path from Sun surface to Earth. The probability of observing a neutrino of flavour $\alpha$ from a mass eigenstate $i$ is given by $|U_{\alpha i}^*|^2$, where $U$ is the unitary PMNS mixing matrix. Thus, the probability of a solar neutrino to manifest as flavour $\alpha$ is given by
\begin{equation}
	P^S_{\nu_e \rightarrow \nu_\alpha}(E) = |U_{\alpha i}^*|^2 P^\odot_{\nu_e \rightarrow \nu_i}(E),
\end{equation}
and we assume throughout this paper that repeated indexes are summed.

If the neutrino flux crosses the Earth the probabilities are modified, since the propagation eigenstates in matter differ from the vacuum ones. In general the flavour eigenstate at time $t$ is  obtained by evolving the one at $t=0$ with the appropriate evolutor operator $\mathcal{U}$
\begin{equation}
	\ket{\nu_\alpha, t} = \mathcal{U}_{\alpha \beta}(t) \ket{\nu_\beta,0},
\end{equation}
where we assume a pure flavour (vacuum) eigenstate vacuum at $t=0$,
\begin{equation}
	\ket{\nu_\alpha, 0} = U_{\alpha i}^* \ket{\nu_i, 0}.
\end{equation}
Notice that the propagation of a single neutrino within Earth is given by the \emph{coherent} propagation of mass eigenstates. The determination of the evolutor operator $\mathcal{U}$ is in general a non-trivial problem, and will be discussed in detail in the following sections; for the moment let us assume we know a closed form expression for it.

A general neutrino state $\ket{\nu}$ can be expressed as a linear superposition of flavour or mass eigenstates. Working in the flavour basis
\begin{equation}
	\ket{\nu, t = 0} = c_\alpha (t=0) \ket{\nu_\alpha, t=0},
\end{equation}
where $\ket{\nu_\alpha, 0}$ stands for a vacuum flavour eigenstate. This state will evolve as
\begin{equation}
	\ket{\nu, t} = c_\alpha(0) \mathcal{U}_{\alpha \beta} \ket{\nu_\beta, 0},
\end{equation}
which allows to define the linear coefficients at time $t$
\begin{equation}
	c_\alpha (t) = \braket{\nu_\alpha, 0}{\nu,t} =  \mathcal{U}_{\alpha \beta}^T (t)\ c_\beta(0),
\end{equation}
coinciding with the probability amplitude of the state $\ket{\nu,t}$ to interact as $\alpha$ flavour at time $t$.

A mass eigenstate expressed as linear combination of flavour eigenstates is
\begin{equation}
	\ket{\nu_i, 0} = U_{i \alpha}^T \ket{\nu_\alpha, 0} = U_{\alpha i} \ket{\nu_\alpha,0},
\end{equation}
which implies a transition amplitude from mass to flavour eigenstate
\begin{equation}
	\braket{\nu_\alpha, 0}{\nu_i, t} = \mathcal{U}_{\alpha \beta}^T (t) U_{\beta i}.
\end{equation}

Putting everything together, the final probability for a solar neutrino to manifest as $\alpha$ flavour while crossing the Earth is given by
\begin{equation}\label{eq:sun_earth_probability}
	P_\alpha^{SE}(t, E) = \left| \mathcal{U}_{\alpha \beta}^T(t) U_{\beta i} \right|^2  P_{\nu_e \rightarrow \nu_i}^\odot(E),
\end{equation}
where $t=0$ is defined at the neutrino crossing the Earth surface.
The interpretation of Eq.~\ref{eq:sun_earth_probability} is the following: $U_{\beta i}$ are the coefficients of the mass eigenstate $i$ expressed as linear combination of flavour eigenstates, $\ket{\nu_i, 0} = U_{\beta i} \ket{\nu_\beta, 0}$, and $\mathcal{U}^T_{\alpha \beta}(t) U_{\beta i} = \braket{\nu_\alpha, 0}{\nu_i, t}$ is the transition amplitude from the evolved mass eigenstate $i$ to interaction eigenstate $\alpha$. Finally, each probability $|\braket{\nu_\alpha, 0}{\nu_i, t}|^2$ is multiplied by the weight of the mass eigenstate $i$ in the incoherent solar flux, $P_{\nu_e \rightarrow \nu_i}^\odot(E)$.

%From this relation the probability amplitude for a neutrino mass eigenstate produced at $t=0$ of interacting as a flavour eigenstate at time $t$ is given by
%\begin{eqnarray}
%	P_{i \rightarrow \alpha}(t) &=& \braket{\nu_\alpha, 0}{\nu_i,t} = \bra{\nu_\alpha, 0} U^T_{i \beta} \ \mathcal{U}_{\beta \gamma}(t) \ket{\nu_\alpha,0} \\ 
%	&=& U^T_{i\beta} \ \mathcal{U}_{\beta \alpha}(t) = \left( U^T \mathcal{U}(t) \right)_{i\alpha}
%\end{eqnarray}

\subsection{Evolutor equations}

The evolutor operator is defined by the equations
\begin{equation}
	\ket{\nu_\alpha, t} = \mathcal{U}_{\alpha \beta}(t) \ket{\nu_\beta, 0}, \hspace{1cm} \text{ with } \hspace{1cm} \mathcal{U}(0) = \mathbb{1}.
\end{equation}

It is possible to define an effective Hamiltonian $H$ for neutrinos~\cite{Fantini:2018itu}, such that the flavour states obey to the Schr{\"o}dinger equation
\begin{equation}
	\frac{\de}{\de t} \ket{\nu_\alpha, t}  = - i H_{\alpha \beta}(t) \ket{\nu_\beta, t},
\end{equation}
giving the differential equation for $\mathcal{U}$
\begin{eqnarray}
 \frac{d}{\de t} \mathcal{U}_{\alpha \beta} (t) \ket{\nu_\beta, 0} &=& -i H_{\alpha \gamma}(t)\ \mathcal{U}_{\gamma \beta}(t) \ket{\nu_\beta, 0} \\
	\Rightarrow \hspace{0.2cm} \frac{\de}{\de t} \mathcal{U}(t) &=& - i H(t)\ \mathcal{U}(t), \label{eq:evolutor}
\end{eqnarray}
where the second equation follows from the states $\ket{\nu_\alpha,0}$ forming a complete orthonormal basis.

Eq.~\ref{eq:evolutor} allows to express the evolutor as a function of the Hamiltonian of the system. Its formal solution is
\begin{equation}
	\mathcal{U}(t) = \mathcal{T} \left[ e^{-i \int_{t_0}^t \de t' H(t')} \right].\label{eq:evolutor_formal}
\end{equation}
where $\mathcal{T}$ is the time-order operator. Eq.~\ref{eq:evolutor_formal} does not generally admit an analytic closed form, except for very special cases, for instance if the Hamiltonian at different times does commute.

A well known approach to the problem is the Dyson series 
\begin{equation}
	\mathcal{U}(t) = \mathbb{1} + \sum_{n=1}^\infty \frac{\left(-i\right)^n}{n!} \int_{t_0}^t \de t_1 \int_{t_0}^t \de t_2 \cdots \int_{t_0}^t \de t_n \mathcal{T} \left[H(t_1) H(t_2) \cdots H(t_n) \right], \label{eq:Dyson_series}
\end{equation}
which allows for an approximate solution obtained by truncating Eq.~\ref{eq:Dyson_series} at finite values of $n$, if the series is expected to be perturbative. 
%However, by adopting this approach the approximate evolutor operator is not longer a unitary operator.
%
%An alternative approximate solution, that we adopt here, is the Magnus expansion~\cite{magnus1954exponential} (ME) (see also~\cite{blanes2009magnus} for a modern review).
%Given the matrix differential equation
%\begin{equation}
%	\frac{\de}{\de t} Y(t) = A(t) Y(t), \hspace{1cm} \text{with} \hspace{1cm} Y(0) = Y_0,
%\end{equation}
%the solution at time $t$ can be expressed as
%\begin{equation}
%	Y(t) = e^{\Omega(t)} Y_0, \hspace{1cm} \text{with} \hspace{1cm} \Omega(t) = \sum_{k=1}^\infty \Omega_k (t). \label{eq:magnus_formal}
%\end{equation}
%The operators $\Omega_k(t)$ are formally defined for each value of $k$. As with the Dyson series, Eq.~\ref{eq:magnus_formal} gives a formal solution, but for practical applications the series must be truncated at finite values of $k$. Nevertheless, the approximate evolutor operator is unitary in the ME.
%
%The first two terms in the ME are
%\begin{eqnarray}
%	\Omega_1(t) &=& \int_0^t \de t_1 A(t_1), \\
%	\Omega_2 (t) &=& \frac{1}{2} \int_0^t \de t_1 \int_0^{t_1}  \de t_2 \left[ A(t_1), A (t_2) \right].
%\end{eqnarray} 
%$\Omega_1$ would give the exact solution if $A$ is constant, such that its commutator at different times vanish. For the general case, the higher-order terms in the ME take into account the $A(t)$ time dependency.

\subsection{Earth matter regeneration}
In the adiabatic regime solar neutrinos leave the Sun as propagation eigenstates $\ket{\nu_2}$. They thus do not oscillate while travelling to the Earth. However, before arriving at a terrestrial detector they can cross the Earth itself, along which path the matter potential makes the propagation eigenstates different from the vacuum ones. This results in neutrino oscillations inside the Earth that, on average, result in a regeneration of $\nu_e$ with respect to the vacuum case.

The electron density inside Earth can be parametrised with 5 shells, within which the electron density varies smoothly as~\cite{Lisi:1997yc}
\begin{equation}\label{eq:earth_density_param}
	N_j(r) = \alpha_j + \beta_j r^2 + \gamma_j r^4, \hspace{1cm} \text{with} \hspace{1cm} \left[N\right] = \text{mol}/\cm^3,
\end{equation}
and where $r$ is the radial distance normalised to the Earth radius. The numerical value of the parameters is reported in Table~\ref{tab:shell_parameters}.

\begin{table}[ht]
\begin{tabular}{clcccc}
\hline
\hline
$j$& Shell          & $[r_{j-1},\,r_j]$  &$\alpha_j$ &$\beta_j$ &$\gamma_j$\\
\hline
1 & Inner core      &     $[0,\,0.192]$    &   6.099 & $-$4.119 &    0.000 \\
2 & Outer core      &   $[0.192,\,0.546]$  &   5.803 & $-$3.653 & $-$1.086 \\
3 & Lower mantle    &   $[0.546,\,0.895]$  &   3.156 & $-$1.459 &    0.280 \\
4 & Transition Zone &   $[0.895,\,0.937]$  &$-$5.376 &   19.210 &$-$12.520 \\
5 & Upper mantle    &     $[0.937,\,1]$    &  11.540 &$-$20.280 &   10.410 \\
\hline 
\hline
\end{tabular}
\caption{\label{tab:shell_parameters}Values of the parameters for the electron density expressed as $N_j(r) = \alpha_j + \beta_j r^2 + \gamma_j r^4$ with $[N] = \text{mol}/\cm^3$, for each of the Earth internal shell. The radial distance $r$ is normalised to the radius of Earth.}
\end{table}

\begin{figure}[ht]
	\includegraphics[width=\textwidth]{earth_01}
	\caption{Earth section showing the different shells and their radial density evolution. The nadir angle $\eta$, radial distance $r$ and trajectory coordinate $x$ are also schematised. Figure taken from~\cite{Lisi:1997yc}}
	\label{fig:earth_density}	
\end{figure}

The parametrisation in Eq.~\ref{eq:earth_density_param} is valid for radial trajectories, i.e. paths crossing the center of the Earth. For a path forming an angle $\eta$ (nadir) with radial trajectory (cf. Fig.~\ref{fig:earth_density}), the following relations hold
\begin{eqnarray}
	N_j(x) &=& \alpha'_j + \beta'_j x^2 + \gamma'_j x^4, \label{eq:earth_density_nadir} \\
	\alpha'_j &=& \alpha_j + \beta_j \sin^2 \eta + \gamma_j \sin^4\eta, \\
	\beta'_j &=& \beta_j + 2 \gamma_j \sin^2\eta, \\
	\gamma'_j &=& \gamma_j.
\end{eqnarray}

The Earth density profile for different values of the nadir angle $\eta$ is reported in Fig.~\ref{fig:earth_density}.

\begin{figure}[ht]
	\includegraphics[width=0.8\textwidth]{earth_density}
	\caption{Earth density profile for different values of the nadir angle $\eta$, following the parametrisation in~\cite{Lisi:1997yc}.}
	\label{fig:earth_density}
\end{figure}

If the detector is placed underground, we can define a ``detector shell'' at the detector radial distance. In our example SNO was placed $H=2$ km underground, so we can define $r_{det} = r_\text{SNO} = 1 - H /R_\text{E} \equiv 1 - h$, where $R_\text{E}$ is the (not rescaled) Earth radius, $R_E = 6.371 \cdot 10^3$ km, cf. Figure~\ref{fig:scheme_detector_shell}.

\begin{figure}[ht]
	\includegraphics[width=0.8\textwidth]{earth_SNO_scheme}
	\caption{Schematic representation of Earth shells and ``detector shell'' for an underground detector.}
	\label{fig:scheme_detector_shell}
\end{figure} 

In general, a shell $i$ is crossed by a neutrino trajectory with nadir angle $\eta$ if $r_i > \sin \eta$. The value of the trajectory coordinate at each shell crossing is given by 
\begin{equation}
	x_i = \sqrt{r_i^2 - \sin^2 \eta}, \hspace{1cm} \text{ for $i$ such that} \hspace{1cm} r_i > \sin \eta.
\end{equation}

For an underground detector the contribution to the trajectory from the outer layer (between Earth surface and detector shell) is given by
\begin{equation}
	 \Delta x(\eta) = \left\{ \begin{array}{l c c}
		-r_{det} \cos\eta + \sqrt{1 - r_{det}^2 \sin^2 \eta} & \text{for} & 0 \leq \eta \leq \frac{\pi}{2}, \\
		r_{det} \cos\eta + \sqrt{1 - r_{det}^2 \sin^2 \eta} & \text{for} & \frac{\pi}{2} \leq \eta \leq \pi.
	\end{array} \right.
\end{equation}
$\Delta x$ reduces to $h$ for $\eta=0, \pi$, and has a maximum at $\sqrt{h(2-h)}$ for $\eta=\pi/2$ (this is approximately 160 km for SNO).

%In the neutrino oscillation dynamics the starting point is Eq.~\ref{eq:evolutor}, with ${\mathcal{U}(0) = \mathbb{1}}$. The ME of the evolutor is thus
%\begin{equation}
%	\mathcal{U}(t) = e^{\Omega(t)},\hspace{1cm} \text{with} \hspace{1cm} \Omega(t) = \sum_{k=1}^\infty \Omega_k (t)
%\end{equation}
%\begin{eqnarray}
%	\Omega_1(t) &=& -i \int_0^t \de t' H(t'), \label{eq:ME1}\\
%	\Omega_2(t) &=& -\frac{1}{2} \int_0^t \de t_1 \int_0^{t_1} \de t_2 \left[ H(t_1), H(t_2) \right]. \label{eq:ME2}
%\end{eqnarray}
%We can verify that, being $\Omega_{1,2}(t)$ anti-hermitian, the truncated evolutor is a unitary operator.
%
%Since solar neutrinos are ultra-relativistic, we can use the distance traveled $r$ within a time $t$ as evolution variable. We split the hamiltonian as

\section{Neutrino propagation Hamiltonian}
We follow in this section the notation of~\cite{Fantini:2018itu}.
The propagation Hamiltonian for an ultrarelativistic neutrino propagating in a medium with electron density $n_e(x)$ is, in the flavour basis
\begin{equation}\label{eq:hamiltonian}
	H_\nu = U^* \text{diag}(k) U^T + V(x) \text{diag}(1,0,0),
\end{equation}
with
\begin{eqnarray}
	k_i &=& \frac{m_i^2}{2E},\\
	V(x) &=& \sqrt{2}G_F n_e(x).
\end{eqnarray}
For antineutrinos, the same Eq.~\ref{eq:hamiltonian} holds, with the replacements
\begin{eqnarray}
	U &\rightarrow U^*,\\
	V &\rightarrow -V.
\end{eqnarray}

For the sake of statistical analysis it is more convenient to redefine the Hamiltonian by subtracting a constant term
\begin{equation}
	H_\nu \rightarrow H_\nu - U^* \left(k_j \mathbb{1}\right) U^T,
\end{equation}
with $j=1$ for normal ordering (NO) and $j=2$ for inverted ordering (IO), such that
\begin{eqnarray}
	\left(\begin{array}{ccc}
k_1 &&\\
& k_2 & \\
&& k_3 	
\end{array} \right) - k_1 \mathbb{1} &=& \frac{1}{2E} \left( \begin{array}{ccc}
 0 && \\
 & \Delta m_{21}^2 & \\
 && \Delta m_{31}^2	
 \end{array} \right) \hspace{0.2cm} \text{ for NO} \label{eq:kinetic_NO}\\
	\left(\begin{array}{ccc}
k_1 &&\\
& k_2 & \\
&& k_3 	
\end{array} \right) - k_2 \mathbb{1} &=& \frac{1}{2E} \left( \begin{array}{ccc}
 - \Delta m_{21}^2 && \\
 & 0 & \\
 && \Delta m_{32}^2	
 \end{array} \right) \hspace{0.2cm} \text{ for IO} \label{eq:kinetic_IO}
\end{eqnarray}
so that we can use as free parameters $\Delta m_{21}^2 >0$ and $\Delta m_{3\ell}^2$, with $\ell=1$ for NO ($\Delta m_{31}^2 > 0$) and $\ell=2$ for IO ($\Delta m_{32}^2 < 0$). In the following we keep using the notation $\text{diag}(k)$ for the general expressions valid for any choice of mass ordering.

The mixing matrix $U$ can be expressed as
\begin{equation}\label{eq:pmns}
	U = R_{23} \Delta R_{13} \Delta^* R_{12},
\end{equation}
where
\begin{equation}\label{eq:PMNS_matrices}
	\begin{array}{ll}
		R_{23} = \left( \begin{array}{ccc}
		1 & 0 & 0 \\
		0 & c_{23} & s_{23} \\
		0 & -s_{23} & c_{23}
	\end{array} \right), &
	R_{13} = \left( \begin{array}{ccc}
		c_{13} & 0 & s_{13}\\
		0 & 1 & 0 \\
		-s_{13} & 0 & c_{13} 
	\end{array} \right), \\
			R_{12} = \left( \begin{array}{ccc}
		c_{12} & s_{12} & 0 \\
		-s_{12} & c_{12} & 0 \\
		0 & 0 & 1
	\end{array} \right), &
	\Delta = \left( \begin{array}{ccc}
		1 & 0 & 0\\
		0 & 1 & 0 \\
		0 & 0 & e^{i\delta} 
	\end{array} \right).
	\end{array}
\end{equation}
With the parametrisation Eq.~\ref{eq:pmns} and using $\left[\Delta,R_{12}\right] = \left[V,R_{23}\right]=\left[V,\Delta\right]=0$, the propagation Hamiltonian can be rewritten as
\begin{equation}
	H_\nu = R_{23} \Delta^* \tilde{H} \Delta R_{23}^T,
\end{equation}
with
\begin{equation}
	\tilde{H} = R_{13} R_{12} \text{diag}(k) R_{12}^T R_{13}^T + V(x) \text{diag}(1,0,0).
\end{equation}
Notice that $\tilde{H}$ does not depend on $\theta_{23}$ nor $\delta$.

Given that $R_{23}$ and $\Delta$ do not depend on position, they can be factorised in the time-ordered definition of the evolutor operator
\begin{eqnarray}
	\mathcal{U} &=& \mathcal{T} e^{-i \int \de x H_\nu (x)} = \mathbb{1} + R_{23} \Delta^* \mathcal{T}\left[-i \int \de x \tilde{H}_\nu (x)\right] \Delta R_{23}^T +\\
	&& \frac{1}{2} R_{23} \Delta^* \mathcal{T}\left[ \left(\left(-i\right)^2 \int \de x_1 \de x_2 \tilde{H}_\nu (x_1) \underbrace{\Delta R_{23}^T R_{23} \Delta^*}_{= \mathbb{1}}  \tilde{H}_\nu (x_2)\right)\right] \Delta R_{23}^T \\ 
	&&  + \dots (\text{ other terms of the Dyson series for } n \rightarrow \infty ) \\
	&=& R_{23} \Delta^* \mathcal{T} \left[ e^{- i \int \de x \tilde{H}(x)} \right] \Delta R_{23}^T. \label{eq:evolutor_factor}\label{eq:evolutor_factorised}
\end{eqnarray}

\subsection{Perturbative expansion of the neutrino propagation Hamiltonian}
We are interested in an expression for the operator in Eq.~\ref{eq:evolutor_factor}
\begin{equation}\label{eq:evolutor_12-13}
	\tilde{\mathcal{U}} = \mathcal{T} \left[ e^{- i \int \de l \tilde{H}(l)} \right],
\end{equation}
where $l$ is the coordinate along the neutrino path. We normalise distances to the Earth radius $R_E$, by defining $x = \l/R_E$. The Hamiltonian $\tilde{H}$ can be divided in a kinetic and a matter dependent terms
\begin{equation}
	\tilde{H}(x) = \tilde{H}_k + \sqrt{2} G_F n_e(x)\text{diag}(1,0,0),
\end{equation}
where $\tilde{H}_k$ does not depend on $x$.

To work out a perturbative expression for $\mathcal{U}$, it is convenient to express the electron density as perturbation along its mean value along the path~\cite{Lisi:1997yc}
\begin{equation}
	n_e(x) = \bar{n}_e + \delta n(x), \hspace{1cm} \bar{n}_e =\frac{1}{x_2-x_1} \int_{x_1}^{x_2} \de x \ n_e(x).
\end{equation}
from which it follows
\begin{equation}
	\int_{x_1}^{x_2} \de x\ \delta n(x) = 0.
\end{equation} 
We can analogously divide the Hamiltonian into a zeroth order term and a perturbation
\begin{equation}
	\tilde{H}(x) = \underbrace{\tilde{H}_k + \sqrt{2}G_F \bar{n}_e \text{diag}(1,0,0)}_{\tilde{H}_0} + \underbrace{\sqrt{2}G_F \delta n(x) \text{diag}(1,0,0)}_{\delta \tilde{H}(x)},
\end{equation}
where again $\tilde{H}_0$ does not depend on $x$.

The evolutor can thus be expressed as~\cite{Lisi:1997yc}
\begin{equation}
	\mathcal{U}(x_2,x_1) = \bar{\mathcal{U}}(x_2,x_1) - i \int_{x_1}^{x_2} \de x\ \bar{\mathcal{U}}(x_2,x)\ \delta \tilde{H}(x)\ \bar{\mathcal{U}}(x,x_1) + \mathcal{O}(\delta\tilde{H}^2),\label{eq:evolutor_1}
\end{equation}
where $\bar{\mathcal{U}}$ is the evolutor for constant matter density $\bar{n}_e$.

In the mass basis,
\begin{equation}\label{eq:Hmass}
	\bar{H}_m = \text{diag}(k_1,k_2,k_3) + U^T \text{diag}(\sqrt{2}G_F \bar{n}_e, 0, 0) U,
\end{equation}
the evolutor for a constant Hamiltonian can generally be expressed in a closed form~\cite{Ohlsson:1999xb}
\begin{equation}
	e^{- i \bar{H}_m x} = \phi \sum_{a=1}^3 e^{-i x \lambda_a} \frac{1}{3\lambda_a^2 + c_1}\left[ \left(\lambda_a^2 + c_1\right) \mathbb{1} + \lambda_a T + T^2 \right] \equiv \phi \sum_{a=1}^3 e^{-i x \lambda_a} M_a,
\end{equation}
where $T = \bar{H}_m - \text{Tr}(\bar{H}_m) \mathbb{1}/3 $ is a traceless matrix. $\lambda_a$ are the roots of the characteristic equation
\begin{eqnarray}
\lambda_1 &=& - \sqrt{-\frac{1}{3} c_1} \cos \left[ \frac{1}{3}
\arctan \left( \frac{1}{c_0} \sqrt{-c_0^2 - \frac{4}{27} c_1^3}
\right) \right] + \sqrt{-c_1} \sin \left[ \frac{1}{3} \arctan \left(
\frac{1}{c_0} \sqrt{-c_0^2 - \frac{4}{27} c_1^3} \right) \right],\\
\lambda_2 &=& - \sqrt{-\frac{1}{3} c_1} \cos \left[ \frac{1}{3} \arctan
\left( \frac{1}{c_0} \sqrt{-c_0^2 - \frac{4}{27} c_1^3} \right)
\right] - \sqrt{-c_1} \sin \left[ \frac{1}{3} \arctan \left(
\frac{1}{c_0} \sqrt{-c_0^2 - \frac{4}{27} c_1^3} \right) \right], \\
\lambda_3 &=& 2 \sqrt{-\frac{1}{3} c_1} \cos \left[ \frac{1}{3} \arctan
\left( \frac{1}{c_0} \sqrt{-c_0^2 - \frac{4}{27} c_1^3} \right)
\right]
\end{eqnarray}
with
\begin{eqnarray}
	c_1 &=& T_{11} T_{22} - T_{12} T_{21} + T_{11} T_{33} - T_{13} T_{31}
+ T_{22} T_{33} - T_{23} T_{32}, \\
c_0 &=& - \det T.
\end{eqnarray}
Finally
\begin{equation}
	\phi = e^{- i x \frac{\text{Tr}(\bar{H}_m)}{3} }.
\end{equation}

In defining $\bar{H}_m$ in Eq.~\ref{eq:Hmass} we assumed the mixing matrix to be real, $U = U^*$, since the effect of CP-violation in neutrino oscillations can be factorised as in Eq.~\ref{eq:evolutor_factorised}.

By noticing that the full dependence on $x$ in $e^{-i \bar{H}_m x}$ is now contained within the scalar functions $e^{- i \lambda_a x}$, the first order correction in Eq.~\ref{eq:evolutor_1} can be computed as
\begin{eqnarray}
\mathcal{U}^{(1)}_m(x_2,x_1) &=&	- i \int_{x_1}^{x_2} \de x\ \bar{\mathcal{U}}(x_2,x)\ \delta \tilde{H}(x)\ \bar{\mathcal{U}}(x,x_1) \\
&=& - i \sum_{a,b=1}^3 \int_{x_1}^{x_2} \de x e^{- i \tilde{\lambda}_a (x_2-x)} M_a U^T \text{diag}\left(\sqrt{2} G_F \delta n(x),0,0\right) U M_b e^{- i \tilde{\lambda}_b (x-x_1)}\\
&=& - i \sum_{a,b=1}^3 M_a U^T \text{diag}\left(\sqrt{2}G_F I_{ab}(x_2, x_1), 0, 0\right) U M_b,
\end{eqnarray}
where  $\tilde{\lambda}_a = \lambda_a + \text{Tr}(\bar{H}_m)/3$ and we defined
\begin{equation}
	I_{ab}(x_2,x_1) = \int_{x_1}^{x_2}\de x\ e^{- i \tilde{\lambda}_a (x_2-x)}\ \delta n(x)\ e^{- i \tilde{\lambda}_b (x-x_1)}.
\end{equation}
For a path fully contained within one shell, 
\begin{equation}
	\delta n (x) = \tilde{\alpha}' + \beta' x^2 + \gamma' x^4,
\end{equation}
where $\tilde{\alpha}' = \alpha' - \bar{n}_e$, and $I_{ab}$ can be expressed analytically in closed form.
Finally, we rotate back to flavour basis
\begin{equation}
	\mathcal{U}_f(x_2,x_1) = U\ \mathcal{U}_m(x_2,x_1)\ U^T,
\end{equation}
obtaining
\begin{eqnarray}
\mathcal{U}_f(x_2,x_1) &=& \mathcal{U}^{(0)}_f(x_2,x_1) +\mathcal{U}^{(1)}_f(x_2,x_1) + \mathcal{O}(\delta \tilde{H}^2),\\
\mathcal{U}^{(0)}_f(x_2,x_1) &=& 	e^{- i \bar{H}_f (x_2-x_1)} = \phi \sum_{a=1}^3 e^{-i (x_2-x_1) \lambda_a} U M_a U^T = \phi \sum_{a=1}^3 e^{-i (x_2-x_1) \lambda_a} \tilde{M}_a,\\
\mathcal{U}^{(1)}_f(x_2,x_1) &=& - i \sum_{a,b=1}^3 \tilde{M}_a\ \text{diag}\left(\sqrt{2}G_F I_{ab}(x_2, x_1), 0, 0\right)\ \tilde{M}_b,
\end{eqnarray}
with $\tilde{M}_a = U M_a U^T$. In our code we implement these perturbative expressions to compute $\tilde{\mathcal{U}}$ in Eq.~\ref{eq:evolutor_12-13}, thus using a ``reduced'' mixing matrix $U = R_{13} R_{12}$, and subsequently use Eq.~\ref{eq:evolutor_factor} to obtain the full dynamics.

%We can now calculate the first terms in the ME. To implement $\Omega_1$ in Eq.~\ref{eq:ME1}
%\begin{eqnarray}
%	\int_{l_1}^{l_2} \de l' \tilde{H}(l') &=& R_E \int_{x_1}^{x_2} \de x' \tilde{H}(x') = R_E \tilde{H}_0 \left(x_2 - x_1 \right) \label{eq:ME1}\\
%	&& + \sqrt{2} G_F R_E \left( \alpha' \left( x_2 - x_1 \right) + \frac{\beta'}{3} \left(x_2^3 - x_1^3\right) + \frac{\gamma'}{5} \left(x_2^5 - x_1^5\right) \right),\nonumber
%\end{eqnarray}
%where we assume the integration path is fully contained within a single Earth shell, so that the parametrisation Eq.~\ref{eq:earth_density_nadir} applies. We will discuss later on how the transition between shells is accounted for.
%
%For the second order term $\Omega_2$ in Eq.~\ref{eq:ME2}
%\begin{equation}
%	R_E^2 \int_{x_1}^{x_2} \de x \int_{x_1}^x \de y \left[ \tilde{H}(x), \tilde{H}(y) \right] = R_E^2 \tilde{H}^{(2)} \int_{x_1}^{x_2} \de x \int_{x_1}^x \de y \left(h(x) - h(y) \right),
%\end{equation}
%where $\tilde{H}^{(2)}$ is the constant matrix with elements
%\begin{eqnarray}
%\tilde{H}^{(2)}_{12} &=& \left(k_2 - k_1\right) c_{12} c_{13} s_{12} = -\tilde{H}^{(2)}_{21}, \\
%\tilde{H}^{(2)}_{13} &=& \frac{1}{2} s_{13} c_{13} \left( \left(k_2-k_1 \right) \left(2 c_{12}^2 - 1 \right) + 2k_3 -k_1 - k_2 \right) = -\tilde{H}^{(2)}_{31},\\
%\tilde{H}^{(2)}_{23} &=& \tilde{H}^{(2)}_{32} = \tilde{H}^{(2)}_{ii} = 0.
%\end{eqnarray}
%The expressions specific for NO and IO can be easily obtained by means of the replacements Eq.s~\ref{eq:kinetic_NO},~\ref{eq:kinetic_IO}.
%
%Assuming again that the path is fully contained within one shell, the integral gives
%\begin{equation}\label{eq:ME2}
%\int_{x_1}^{x_2} \de x \int_{x_1}^x \de y \left( h(x) - h(y) \right) = \frac{1}{30} \left( x_2 - x_1 \right)^3 \left( x_1 + x_2 \right)\left[ 5 \beta' + 2 \gamma' \left( 2 x_1^2 +  2x_2^2 + x_1 x_2 \right) \right]
%\end{equation}
This procedure allows to express the evolutor at 1st order in $\delta H$, for a path fully contained within one shell. For paths crossing multiple shells, it can be noticed~\cite{Lisi:1997yc} that the full evolutor on a path $(x_1,x_2)$ can be expressed as a time-ordered product of evolutors along the same path
\begin{equation}\label{eq:evolutor_product}
	\mathcal{U}(x_2, x_1) = \mathcal{U}(x_2, x_i) \mathcal{U}(x_i, x_1),
\end{equation}
where $x_i$ is a generic point $x_1 < x_i <x_2$ contained on the original path. Moreover, it can be shown~\cite{Lisi:1997yc} that
\begin{equation}\label{eq:evolutor_transpose}
	\mathcal{U}(0, -x) = \mathcal{U}(x,0)^T.
\end{equation}
The consequences of Eq.s~\ref{eq:evolutor_product},~\ref{eq:evolutor_transpose} are twofold: first, for a path starting at $x=x_0$, with $0\leq  x_0 < x_1$,  crossing $n$ shells with boundaries at trajectory coordinate $(0, x_1, x_2, \dots, x_n)$, and ending at the point $x=x_f$, with $x_{n-1} < x_f \leq x_n$, the full evolutor can be expressed as
\begin{equation}
	\mathcal{U}(x_f, x_0) = \mathcal{U}(x_f, x_{n-1})\mathcal{U}(x_{n-1},x_{n-2})\dots \mathcal{U}(x_2, x_1)\mathcal{U}(x_1, x_0).
\end{equation}
Second, for detectors placed at surface, the Earth spherical symmetry implies that the electron density is symmetric with respect to the trajectory midpoint at $x=0$; thus we only need to compute the evolutor on one half-path (cf. Fig.~\ref{fig:earth_density})
\begin{equation}
	\mathcal{U}(x_F,x_I) = \mathcal{U}(x_F, 0) \mathcal{U}(0, - x_F) = \mathcal{U}(x_F, 0) \mathcal{U}(x_F, 0)^T.
\end{equation}
Notice that the final evolutor is only a function of $\eta$, since both the density profile and travelled distance within Earth are a function of the nadir angle.

If the detector is placed underground at trajectory coordinate $x_{det} < x_F$, we distinguish two cases. For $0 \leq \eta < \pi/2$
\begin{equation}
	\mathcal{U}(x_{det},x_I) = \mathcal{U}(x_{det}, 0) \mathcal{U}(0, - x_F) = \mathcal{U}(x_{det}, 0) \mathcal{U}(x_F, 0)^T, \hspace{0.2cm} \left(0 \leq \eta < \frac{\pi}{2} \right)
\end{equation}
where $x_{det} = \sqrt{r_{det}^2 - \sin\eta^2}$. For $\pi/2 \leq \eta \leq \pi$ the electron density can be approximated to a constant value, since the neutrino path is never deeper than $H$ and, for $H = 2$ km,
\begin{equation}
	\frac{N(R_E) - N(R_E-H)}{N(R_E) + N(R_E - H)} = 10^{-4}
\end{equation}
We fix for simplicity the electron density value to the one at Earth radius, $n_1 = 1.67$ mol/cm${^3}$. Having assumed constant density along the path, the evolutor is simply given by
\begin{equation}
	\tilde{\mathcal{U}}(\eta) = e^{-i R_E \Delta x(\eta) \left(\tilde{H}_0 + \text{diag}\left(\sqrt{2}G_F n_1, 0, 0\right) \right)}, \hspace{0.5cm} \left(\frac{\pi}{2} \leq \eta < \pi \right).
\end{equation}

\section{Time integration}
Solar neutrino experiments typically collect data over a finite interval of time. As such, the measured survival probability is averaged over exposure
\begin{equation}\label{eq:time_integral_general}
	\left< P_E \right> = \frac{\int_{\tau_{d_1}}^{\tau_{d_2}} \de \tau_d \int_{\tau_{h_1} (\tau_d)}^{\tau_{h_2} (\tau_d)} \de \tau_h P_E \left(\eta(\tau_d, \tau_h) \right)}{\int_{\tau_{d_1}}^{\tau_{d_2}} \de \tau_d \int_{\tau_{h_1} (\tau_d)}^{\tau_{h_2} (\tau_d)} \de \tau_h },
\end{equation}

where $\tau_d$ is the daily time, $\tau_h$ the hourly time and $\eta$ the Sun nadir angle at detector location. The integration in Eq.~\ref{eq:time_integral_general} is typically not the most convenient choice for practical applications, a more effective option is to transform the double integral into a single one over $\eta$~\cite{Lisi:1997yc}
\begin{equation}
	\left< P_E \right> = \int_{0}^{\pi} \de \eta W(\eta) P_E(\eta),
\end{equation}
where $W(\eta)$ is a normalized weight function representing the fraction of time the experiment collected data at nadir angle $\eta$.
For real experiments $W(\eta)$ must be provided by the collaboration, taking into account the actual times in which the detector collected data or has been offline. It is nevertheless possible to compute $W(\eta)$ analytically, for the ideal case of an experiment continuously taking data between days $\tau_{d_1}$ and $\tau_{d_2}$~\cite{Lisi:1997yc}.
This is done by changing integration variables
\begin{eqnarray}
	\int_{\tau_{d_1}}^{\tau_{d_2}} \de \tau_d \int_{\tau_{h_1} (\tau_d)}^{\tau_{h_2} (\tau_d)} \de \tau_h P_E \left(\eta(\tau_d, \tau_h) \right) &=& \int_{\tau_{d_1}}^{\tau_{d_2}} \de \tau_d \int_{0}^{\pi} \de \eta \frac{\de \tau_h (\tau_d, \eta)}{\de \eta} P_E \left(\eta \right) \\
	= \int_{0}^{\pi} \de \eta  P_E \left(\eta \right) \int_{\tau_{d_1}}^{\tau_{d_2}} \de \tau_d \frac{\de \tau_h (\tau_d, \eta)}{\de \eta} &=& \int_{0}^{\pi} \de \eta P_E(\eta) W(\eta) \label{eq:integral_W}
\end{eqnarray}

By normalising the daily and hourly times to the interval $[0, 2 \pi]$
\begin{equation}
	\tau_d = \frac{\text{day}}{365} 2\pi, \hspace{2cm} \tau_h = \frac{\text{hour}}{24} 2\pi,
\end{equation}
with $\tau_d=0$ at winter solstice and $\tau_h=0$ at mid-night, it is possible to express
\begin{equation}
	\tau_h = \arccos\left(\frac{\sin(\lambda) \sin(\delta_S) + \cos(\eta)}{\cos(\lambda) \cos(\delta_S)} \right),
\end{equation}
with $\lambda$ the detector latitude and $\delta_S$ the Sun declination, given by
\begin{equation}
	\delta_S = \arcsin\left(-\sin(i) \cos(\tau_d)\right),
\end{equation}
with $i=0.4091$ rad the Earth inclination. With these definitions it is possible to perform the integral defining $W(\eta)$ in in Eq.~\ref{eq:integral_W}; it is convenient to restrict $\tau_d$ within the interval $[0, \pi]$ (the alternative case can be easily derived from this one by using the symmetry of the orbit) and to change integration variable from $\tau_d$ to $T = \cos(\tau_d)$. 
The resulting indefinite integral is expressed in terms of elementary functions and of the incomplete elliptic integral of the first kind; its analytic expression is not particularly illuminating but can be easily evaluated numerically.
Some care must be taken in defining the range of integration for the definite integral, as it is given by the intersection of three intervals: i) $T \in [-1, 1]$ is the interval where $T=\cos(\tau_d)$ is defined, ii) $T \in [\sin(\lambda - \eta)/\sin(i), \sin(\lambda + \eta)/\sin(i)]$ is the range where $T$ can take values for fixed values of $\eta, \lambda, i$, iii) $T \in [\cos(\tau_{d_2}), \cos(\tau_{d_1})]$ is the observation time. If the intersection of the three interval is null then $W(\eta)$ will vanish for that given combination of $\lambda, \eta$ values.

We plot in Fig.~\ref{fig:eta_weights} the weight function for one full year of exposure for three ideal detectors located at latitudes $\lambda = 0, 45^\circ, 89^\circ$.
\begin{figure}[ht]
	\includegraphics[width=0.8\textwidth]{eta_weights.pdf}	
	\caption{Weight of the nadir angle exposure for an ideal experiment located at latitude $\lambda$, taking data continuously over a full year. The coloured regions represent the nadir angles subtending the Earth internal shells as parametrised in Table~\ref{tab:shell_parameters}, for a detector located at the Earth surface.}
	\label{fig:eta_weights}
\end{figure}

\bibliographystyle{JHEP}
\bibliography{bibliography.bib}

\end{document}